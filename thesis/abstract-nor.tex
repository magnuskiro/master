\section*{\Huge Abstrakt \small{Norsk versjon}}
\addcontentsline{toc}{chapter}{Abstrakt - norsk versjon}
$\\[0.1cm]$
\begin{abstract}\label{abstract-nor}

\paragraph{Bakgrunn}
\hspace{0pt}\\
Som Twitter har blitt en global microblogging nettstedet, sin innflytelse i
aksjemarkedet har blitt betydelig. Dette gjør tweets et interessant medium for
samle følelser. En følelse som kan påvirke utviklingen i aksjemarkemarkedet .

\paragraph{motivasjon}
\hspace{0pt}\\
Hvis twitter kan brukes til å forutsi trender i aksjemarkedet den alminnelige
investor ville få en fordel over en dagtrader eller de moderne
handelsalgoritmer. Et annet interessant aspekt er rollen til Twitter i sentiment
analyse. og hvordan Twitters rolle som en datakilde påvirker utviklingen i
aksjemarkedet.

\paragraph{Data og eksperimenter}
\hspace{0pt}\\
Twitter brukes som datakilde. Det gir enkel tilgang, masse data, og mange
muligheter til å bruke tilgjengelige metadata. For å finne den oppfatningen av
en tweet vi bruke pose med ord, SVM, og Naive Bayes. For finans del og
sammenligning av trendene vi bruker lager data fra Oslo Børs. Vi bruker glidende
gjennomsnitt (MA) og gjennomsnittlig retnings indeks (ADX) som trend
indikatorer. Trend sammenligning er basert på MA og ADX. Vi beregner MA og ADX
med finansdataog sentiment data basert på tweets. Deretter sammenligner vi
grafene.

\paragraph{funn}
\hspace{0pt}\\
Vi utforsker bruken av lister over ord, ordbøker, i sentiment analyse. Og vi ser
på innhenting av data fra Twitter og trenden vi kan lage fra det. Med varierende
resultater får vi positive resultater med ordbøkene, mens trend aggregering
mangler finesse og resulterer det burde ha hatt.

\paragraph{Konklusjon}
\hspace{0pt}\\
Sentiment klassifisering av tweets jobbet med begge metodene. Vi klarte å
aggregere en trend basert på følelser. Men sammenligningen med finans trend
fungerte ikke ut som vi håpet. Ingen likheter ble funnet mellom sentiment trend
og finans trend.

\end{abstract}

