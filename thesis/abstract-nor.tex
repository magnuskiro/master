\section*{\Huge Abstrakt \small{Norsk versjon}}
\addcontentsline{toc}{chapter}{Abstrakt - norsk versjon}
$\\[0.5cm]$
\begin{abstract}\label{abstract}

\paragraph{Background:}
\hspace{0pt}\\
As Twitter has become a global microblogging site, its influence in the stock
market has become significant. This makes tweets an interesting medium for
gathering sentiment. A sentiment that might influence trends in the stock
market. 

\paragraph{Motivation:} 
\hspace{0pt}\\
If twitter can be used to predict trends in the stock market
the casual investor would gain an advantage over the day-trader or the modern trading algorithms. 

Another interesting aspect is the role of twitter in sentiment
analysis. And how twitters role as a data source influences trends in the stock
market.   

\paragraph{Methods and experiments:} 
\hspace{0pt}\\
%TODO use the right method descriptions and what I have used.
Twitter is used as the data source. It provides easy access, lots of data, and
many possibilities to utilise the available metadata. 

To improve and verify the sentiment classification and trend comparisons we use
a variation of methods. Simple statistical methods, such as counting positive
and negative words. More advanced methods such as part of speech and other NLP
related magic.%TODO figure out what the magic is. 
We also explore the use of mete data such as location and language tags. 

\paragraph{Results:} 
\hspace{0pt}\\
Rough results of my research. 

\paragraph{Conclusion:} 
\hspace{0pt}\\
All OK ? No? 

\end{abstract}

