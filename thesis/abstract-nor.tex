\section*{\Huge Abstrakt \small{Norwegian}}
\addcontentsline{toc}{chapter}{Abstrakt}
$\\[0.1cm]$
\begin{abstract}\label{abstract-nor}

\paragraph{Bakgrunn}
\hspace{0pt}\\
Siden Twitter har blitt et globlat nettsted for mikroblogging har innflytelsen
til Twitter i aksjemarkedet bitt betydelig. 
Dette gjør tweets til et interessant medium for å samle sentiment. Et sentiment
som potensielt kan påvirke utviklingen i aksjemarkemarkedet.

\paragraph{Motivasjon}
\hspace{0pt}\\
Hvis Twitter kan brukes til å forutsi trender i aksjemarkedet vil den alminnelige
investor få en fordel over en intra-dag trader eller de moderne
trading-algoritmene. Et annet interessant aspekt er rollen til Twitter i sentiment
analyse, og hvordan Twitter sin rolle som datakilde påvirker utviklingen i
aksjemarkedet.

\paragraph{Data og Eksperimenter}
\hspace{0pt}\\
Twitter brukes som datakilde og gir enkel tilgang, mye data, og mange
muligheter til å bruke tilgjengelige metadata. For å finne sentimentet av
en tweet bruke vi 'bag of words', SVM, og Naive Bayes. For finans delen og
sammenligning av trendene bruker vi data fra Oslo Børs. Vi bruker 'moving
average'(MA) og 'average directional index'(ADX) som trend
indikatorer. Trend sammenligningen er basert på MA og ADX. Vi beregner MA og ADX
med finansdata, og sentiment data basert på tweets. Deretter sammenligner vi
grafene.

\paragraph{Funn}
\hspace{0pt}\\
Vi utforsker bruken av lister med ord, ordbøker, i sentiment analyse. Og vi ser
på innhenting av data fra Twitter og trenden vi kan lage med dataene. I
varierende grad får vi positive resultater med ordbøkene, mens trend aggregering
mangler finesse, og kvaliteten den burde ha hatt.

\paragraph{Konklusjon}
\hspace{0pt}\\
Sentiment klassifisering av tweets fungerte med begge metodene. Vi klarte å
aggregere en trend basert på sentiment. Men sammenligningen med finans trendene
fungerte ikke som vi håpet. Ingen likheter ble funnet mellom sentiment trenden
og finans trenden.

\end{abstract}

