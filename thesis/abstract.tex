\section*{\Huge Abstract \small{English}}
\addcontentsline{toc}{chapter}{Abstract}
$\\[0.1cm]$
\begin{abstract}\label{abstract}

\paragraph{Background}
\hspace{0pt}\\
As Twitter has become a global microblogging site, it's influence in the stock
market has become significant. This makes tweets an interesting medium for
gathering sentiment. A sentiment that might influence trends in the stock
market. 

\paragraph{Motivation} 
\hspace{0pt}\\
If Twitter can be used to predict trends in the stock market
the casual investor would gain an advantage over the day-trader or the modern trading algorithms. 

Another interesting aspect is the role of Twitter in sentiment
analysis. And how Twitters role as a data source influences trends in the stock
market.   

\paragraph{Data and Experiments} 
\hspace{0pt}\\
Twitter is used as the data source. It provides easy access, lots of data, and
many possibilities to use available metadata. To find the sentiment of a tweet
we use bag of words, SVM, and Naive Bayes. For the finance part and comparison
of trends we use stock data from Oslo stock exchange. We use moving average(MA)
and average directional index(ADX) as trend indicators. Trend comparison is
based on MA and ADX. We calculate MA and ADX with finance data and sentiment
data based on tweets. Then we compare the graphs. 

\paragraph{Findings}
\hspace{0pt}\\
We explore the usage of lists of words, dictionaries, in sentiment analysis.
And we look at data retrieval from Twitter and the trend we can create from it.
To a varying degree we get positive results with the dictionaries, while the
trend aggregation lacks the finesse and results it should have had.

\paragraph{Conclusion} 
\hspace{0pt}\\
Sentiment classification of tweets worked with both methods. We managed to
aggregate a trend based on sentiment. But the comparison with the finance trend
did not work out as hoped. No similarities was found between the sentiment
trend and the finance trend.  

\end{abstract}

