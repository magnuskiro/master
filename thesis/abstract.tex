\section*{\Huge Abstract \small{English}}
\addcontentsline{toc}{chapter}{Abstract}
$\\[0.1cm]$
\begin{abstract}\label{abstract}

\paragraph{Background:}
As Twitter has become a global microblogging site, it's influence in the stock
market has become noticeable. This makes tweets an interesting medium for
gathering sentiment. A sentiment that might influence trends in the stock
market. 

\paragraph{Motivation:} 
If Twitter can be used to predict future prices in the stock market
the casual investor would gain an advantage over the day-trader and the modern trading algorithms. 

Another interesting aspect is the role of Twitter in sentiment
analysis. And how Twitters role as a data source influences trends in the stock
market.   

\paragraph{Data and Experiments:} 
Twitter is used as the data source. It provides easy access, lots of data, and
many possibilities to use available metadata. 
To find the sentiment of a tweet we use two methods, counting positive and
negative words(bag of words), and classifiers (SVM and Naive Bayes). 
We use moving average(MA) and average directional index(ADX) as trend
indicators. We calculate MA and ADX with data from Oslo stock exchange, and we
created our own indicators, based on MA and ADX, using data from Twitter. Then we
compare the graphs.  

\paragraph{Findings:}
We explore the usage of lists of words, dictionaries, in sentiment analysis.
And we look at data retrieval from Twitter and the trend we can create from it.
To a varying degree we get positive results with the dictionaries, while the
trend aggregation lacks the finesse and results we hoped for.

\paragraph{Conclusion:} 
Sentiment classification of tweets worked with both bag of words, and trained
classifiers. We also managed to aggregate a trend based on sentiment, but we
found no correlation between the financial trend indicators and the sentiment indicators. 

\end{abstract}

