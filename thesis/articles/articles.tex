% header 
\chapter{Processed Articles}

\section{Article template}
file:\textit{filename.pdf} & citation:\cite[]{}  

* What did they use tweets for?\\
* How are tweets used?\\
* Event detection. Is the tweet about merging? \\
* Where can this article be useful later? \\
* What does this article give answers to?\\
% END header 

%done. 

\section{Tweets and Trades: The Information Content of Stock Microblogs}
file:\textit{SSRN-id1702854.pdf} & citation:\cite[]{sprenger10} %todo fix bib and citation ref

% todo add articles to bibliography and create citations. 
* What did they use tweets for?\\
"We find the sentiment (i.e., bullishness) of tweets to be associated with abnormal
stock returns and message volume to predict next-day trading volume."
\cite[]{sprenger10} 

* How are tweets used?\\

* Event detection. Is the tweet about merging? \\

* Where can this article be useful later? \\
What twitter is used for, Twitter chapter. 

Twitter incentives. \cite[p4]{sprenger10}

Description of bullishness, message volume and what it does etc. 

\cite[p52]{sprenger10} suggest that stock microblogs can claim to capture key aspects of the market
conversation.

Picking the right tweets remains just as difficult as making the
right trades.

* What does this article give answers to?\\
Whether bullishness can predict returns.
Whether message volume is related to returns, trading volume, or volatility.
Whether the level of disagreement among messages correlates with trading volume
or volatility.
Whether and to what extent the information content of stock microblogs reflects financial market developments
Whether microblogging forums provide an efficient mechanism to weigh and aggregate information


\section{Exploiting Topic based Twitter Sentiment for Stock Prediction}
file:\textit{filename.pdf} & citation:\cite[]{mukherjee13}  

* What did they use tweets for?\\
Predicting the stock market. 
Stock index time series analysis. 
daily one-day-ahead predictions. 

* How are tweets used?\\
Dirichlet Process mixture model to learn the daily topic set.
Vector regression. 
Topic-based prediction. 

* Event detection. Is the tweet about merging? \\
* Where can this article be useful later? \\
Twitter’s topic based sentiment can improve the prediction accuracy.
\cite[p28]{mukherjee13}

* What does this article give answers to?\\

\section{Twitter as driver of stock price}
file:\textit{Twitter as driver of stock price-Jubbega.pdf} &
citation:\cite[]{annikajubbega11:twitter_driver_stock_price}

* What did they use tweets for?\\
* How are tweets used?\\
* Event detection. Is the tweet about merging? \\
* Where can this article be useful later? \\
General about twitter. 
* What does this article give answers to?\\

\section{Twitter Polarity Classification with Label Propagation over Lexical Links and the Follower Graph}
file:\textit{twitter polarity classification.pdf} & citation:\cite[]{sperious11}

* What did they use tweets for?\\
Polarity classification. Positive/negative. 

* How are tweets used?\\
With label propagation.
Distant supervision. 
Graph based data structure. 
user-->tweet-->bigram/unigram/hashtag/etc.  

* Event detection. Is the tweet about merging? \\
* Where can this article be useful later? \\
Data section / sentiment / 

Twitter section: What people uses twitter for. 

Label propagation approach rivals a model supervised with in-domain annotated tweets and outperforms the noisily supervised classifier and a lexicon-based polarity ratio classifier. \cite[]{sperious11}

Twitter represents one of the largest and most dynamic datasets of user
generated content.


* What does this article give answers to?\\


\section{AVAYA: Sentiment Analysis on Twitter with Self-Training and Polarity Lexicon Expansion}
file:\textit{Sentiment Analysis on Twitter with Self-Training and Polarity
Lexicon Expansion.pdf} & citation:\cite[]{becker13}

* What did they use tweets for?\\
Contextual Polarity Disambiguation and Message Polarity Classification
* How are tweets used?\\
Constrained learning with supervised learning. 
Unconstrained model that used semi-supervised learning in the form of self-training and polarity lexicon expansion

* Event detection. Is the tweet about merging? \\
* Where can this article be useful later? \\
Technical approach of models and sentiment analysis.
State of the art on sentiment analysis with twitter. 

* What does this article give answers to?\\
dependency parses, polarity lexicons,
and unlabeled tweets for sentiment classification on
short messages

We hypothesize this performance is largely due to the expanded vocabulary obtained via unlabeled data and the richer syntactic context captured with dependency path representations. \cite[]{becker13}


\section{Robust Sentiment Detection on Twitter from Biased and Noisy Data}
file:\textit{Robust Sentiment Detection on Twitter from Biased and Noisy
Data.pdf} & citation:\cite[]{barbosa10}

* What did they use tweets for?\\
Sentiment analysis with focus on noise reduction. 

* How are tweets used?\\
Noisy labels. Classifies tweets as subjective or objective. Then distinguishes
the subjective into positive and negative tweets.  
Generalization of tweet classification. Meta-information. How tweets are
written. More abstract representation.

* Where can this article be useful later? \\
Previous work, sentiment analysis, twitter, sentiment features. 
* What does this article give answers to?\\
It provides a better way to classify tweets. 

\section{Investor sentiment and the near-term stock market}
file:\textit{Investor sentiment and the near-term stock market.pdf} & citation:\cite[]{Brown20041}

* Where can this article be useful later? \\
In the finance chapter for historic value and where we have come from. 

\cite[p2]{brown20041} on over-reaction of investors writes: "
He(Siegel (1992)) concludes that shifts in investor sentiment are correlated with market
returns around the crash. Intuitively, sentiment represents the expectations of market participants
relative to a norm: a bullish (bearish) investor expects returns to be above
(below) average, whatever ‘‘average’’ may be.". In the light of resent changes
in the financial world and the utilisation of sentiment from social media, the
notion that opinions and sentiment of investors and market actors affect the
market is not a new observation. 

\cite[p3]{Brown20041} indicates that the sentiment does not cause subsequent
market returns. For a short-term marketing timing this is bad news. However
with the changes in social media over the last decade how is the situation
today? With the microblogging sphere of today we can easily see the
correlation of sentiment and the market indicators [todo:Citation]. But
does the sentiment cause changes in the market-return? 
\cite[p3]{Brown20041} also says that optimism is associated with overvaluation and subsequent low returns.

* What does this article give answers to?\\
\cite[p]{Brown20041} concludes that aggregated sentiment measures has strong
co-movement with changes in the market. He also indicates that sentiment
doesn't appear to be a good trading strategy. This, in the view of
\cite[]{Zhang201155} indicates a leap in sentiment research and what is possible
with the microblogging of today. 

\section{Predicting Stock Market Indicators Through Twitter\\ “I hope it is not as bad as I fear”}
file:\textit{Predicting Stock Market Indicators Through Twitter.pdf} & citation:\cite[]{Zhang201155}

* What did they use tweets for?\\
Gather hope and fear for each day using tweets. 
The sentiment indication of each day is compared to the marked indicators of
the same day. 

* How are tweets used?\\
Get the Positive/negative sentiment. 

* Event detection. Is the tweet about merging? \\
* Where can this article be useful later? \\
Address the question of intention of users on twitter. 
Good summary of things done in regards to twitter. (Might be a bit outdated,
from 2010). 

* What does this article give answers to?\\
That hope, fear and worry makes the stock go down the day after. Calm times,
little hope, fear or worry, makes the stock go up. 


\section{Deriving market intelligence from microblogs}
file:\textit{Deriving market intelligence from microblogs.pdf} & citation:\cite[]{Li2013206}

* How are tweets used?\\
Companies use twitter for feedback and customer relations. Questions can be
asked with a hashtag of to a specific user. This makes it easy to sort filter
the messages, and therefore easier to get in contact with the customer. Best
Buy demonstrated the successfulness of twitter in customer relations by
answering questions with a specific hashtag. In 2009 they had answered nearly
20 thousand questions using twitter. \cite[p1]{Li2013206}
Market Intelligence is also a major aspect of the microbloggin sphere. 

* What did they use tweets for?\\
\cite[]{Li2013206} approaches the classification of sentiment in tweets in four
steps. First is the topic detection. The topic is the overall theme of the
message. This step extracts and identifies the topics associated with the
queries of users. Following that the classification of opinion happens. This
judges the polarity of the sentiment. The state of mind of the user can be
recorded. 

A problem that arises is the credibility of the expresser. This is addressed to
get a better summary of the sentiment. Then \cite[]{Li2013206} aggregates, the
three previously described parts of the classification, to get a truer
reflection of the opinions. 


* Event detection. Is the tweet about merging? \\
* Where can this article be useful later? \\
stateOf-twitter / state-sentiment /  data / 

* What does this article give answers to?\\

\section{The social media stock pickers}
file:\textit{social\_media\_stock\_pickers.pdf} & citation:\cite[]{stevenson12:social_media_stock_pickers}

Opinion mining on the web is not a new phenomenon. But in resent years it has
become much more attractive to traders in the financial world. Twitter and the
social media's opinion is on the rise. This means a surplus of raw data with
easy access. Companies all over the world has started to use twitter and
readily available tweets to their benefit. Trading with social media is part of
the trend. Although there are some drawbacks and shortcomings. Noise and
garbage is one of them. It's difficult to accurately sort through all the data
and get only the information relevant for your use. Even if your right 80\% of
the time, the last 20\% can prove devastating. \cite[]{stevenson12:social_media_stock_pickers}

\section{Sentiment and Momentum}\label{sentiment_and_momentum}
file:\textit{SSRN-id1479197.pdf} & cition:\cite[]{doukas10:sentiment_and_momentum}

Not Twitter. Intra-day transaction data. 
Sentiment affects the profitability of price momentum strategies.

% Possible content.
Use of sentiment can predict changes and momentum in the market.
Bad news in an optimistic period creates cognitive dissonance in the small
investors. This impacts the market by slowing down the selling rate of loosing
stocks. \cite[p29]{doukas10:sentiment_and_momentum} 

Sentiment broadly refers to the state of mind a person has. Whereas negative of
positive. Based on the current state of mind the person will do optimistic or
pessimistic choices. A positive state of mind leads to optimistic judgements of
future events. And a negative state of mind leads to pessimistic judgements. 
\cite[p4]{doukas10:sentiment_and_momentum} 
% todo get citation for psychology articles. 

Further we can see that optimistic sentiment has a 2\% monthly average return. 
While the investor sentiment is pessimistic we see a drastic reduction in
returns. Down to 0.34\%.\cite[p5]{doukas10:sentiment_and_momentum} 
After optimistic periods it is indicated that the monthly return is reduced to
-0.49\%. On the contrary there is no equivalent change after a pessimistic
period. \cite[p6-7]{doukas10:sentiment_and_momentum} 
Momentum profits are only significant when the sentiment is optimistic.
\cite[p29]{doukas10:sentiment_and_momentum}


% todo fix article info for the following articles. 
\section{Is Trading with Twitter only for Twits?}\label{art:ITTT}
Document Description: Blog post that describes the findings of the atricle [todo
art:ref]. 

The article has developed a strategy for trading stocks based on the
bullishness of the tweet. [todo glossary bullishness] Bullishness as I
understand it
is the same as the negativity of the tweet. 

The article bases it's findings on three factors. The holding time of a stock
(the time from you buy it until it's sold). The history of x days (how many of
the past days are used to determine the tweet signal[todo glossary tweet
signal]). And the number of picks (how many stocks you hold a any given
time). 

It is also indicated that The main article has some good information about how
tweets are built up. (Dollar-tagging for representation of a given stock,
\$AAPL)

Has a good figure of the system. 

Indicates that the message volume and trade volume are related. 

RefArticle: \ref{art:TMPSM} Twitter mood Predicts the Stock Market.

Tags: buy/sell-signals, tweet signals, dollar-tagged, OpinionFinder, GPOMS, 

\section{From Tweets to Polls: Linking Text Sentiment to Public Opinion
Time
Series}

The article uses polling data and two years of tweets as their data. 

Basically a comparison of the opinion expressed on twitter and the opinion from
phone enquiries. 

Uses word counting to distinguish relevant tweets from the rest. 

The twitter dataset is huge, typically billions of tweets. 

Daily sentiment = positive tweets / negative tweets.

