\chapter{The Code}\label{code}
The code is complete in the sense of doing what it should. Or making an attempt
to. The results of the coding are up for discussion.

For functionality we have code for mining tweets, compiling dictionaries,
classifying tweets, and aggregating a trend based on data from twitter.  

For this chapter we start of with the structure of the code,
\ref{code:structure}. Then touch the technology and libraries,
\ref{code:technology_libraries}, used up to this point. Before we look at the
data mining, \ref{data:data_retrieval}, dictionary compilation,
\ref{code:dictionary_compilation}, classification,
\ref{code:sentiment_classification}, and trend aggregation,
\ref{code:trend_aggregation}. Last we describe the comparison,
\ref{code:comparison}, of the trends and ending with the issues,
\ref{code:issues}, of the code.  

\section{Structure}\label{code:structure}
There are four main folders. One for classification, one for mining tweets, one
for the dictionaries and on for the files associated with trend. 

Code for classification is split into four files. One file for utilities,
helper functions that does not touch logic. And three for classification:
manual classification, word counting classification, and classification with
classifiers. List\_threshold\_results plots the results from the threshold
variation in graphs.  

The dictionary code is split onto two files, helpers and utility in one, and
logic and execution in the other. 

Trend code is in the trend folder. There we have mining\_utils, which is a
replica of the same file in the Twitter folder. This file only helps with the
acquisition of tweets. The trend\_mining file executes the mining operations,
and acquires tweets from Twitter based on a set of search words. The
trend\_tweet\_sorting file sorts tweets from all the raw search word data files
into files based on date. So all tweets from the same date ends in the same
file. We also have the trend\_compilation file, which contains code for
compiling and plotting trend data and financial data.  

Last we have the 'twitter' directory which has code for extracting tweets from
twitter. This code is used for creating the datasets that the dictionaries are
built from. We have the mining\_utils which is helper code for connection and
writing files. While the mining\_operations file does the logic of
managing the acquired tweets. 

The important files are listed under. '- something' are folders, while the
others are actual Python files. The indentation shows which files are in which
folders.  
\begin{verbatim}
- code    
    - classification
        classification_utils.py
        list_threshold_results.py
        manual_classification.py
        svm_bayes_classification.py
        word_count_classification.py
    - dictionaries
        dictionaries.py
        dictionary_utils.py
    - trend    
        mining_utils.py
        trend_classification_utils.py
        trend_compilation.py
        trend_mining.py
        trend_tweet_sorting.py
        ma_adx_plotter.py  
    - twitter
        mining_operations.py
        mining_utils.py
    graph_plots.py
\end{verbatim}

\section{Technology and Libraries}\label{code:technology_libraries}
The used Python libraries are: ConfigParser, ast, codecs, io, os, twython,
time, matplotlib, urllib, re, nltk, sklearn.
Most of them are standard, while some are not. Some are self
explanatory. 

The libraries that are not shipped with standard Python are described in the
following paragraphs. 

\paragraph{Twython}
Library for connection and integration with the Twitter API. 
The source and documentation can be found on github:
\url{https://github.com/ryanmcgrath/twython}

\paragraph{nltk}
The natural language tool kit (nltk) is a library that provides functionality
for working with human language. It has functionality such as classifiers and
tokenization tools. See more at \url{http://www.nltk.org/}.

\paragraph{sklearn}
Scikit-learn provides functionality for learning algorithms, machine learning
and classification. We use this library to provide the kernels for
classification with classifiers. \url{http://scikit-learn.org/}

\paragraph{matplotlib}
Matplotlib is a library for graph plotting. And that is what we use it for.
\url{http://matplotlib.org/}.
 
\section{Data Retrieval}\label{data:data_retrieval}
As we have two data sources we split it into finance data and data from twitter.

\subsection{Twitter}
The mining operation can be seen as three steps. This happens in
\textit{mining\_operations.py} and \textit{mining\_utils.py}.

\paragraph{Step 1}
\hspace{0pt}\\
Authenticating and connecting to twitter. 
\begin{python}
conf = open('auth.cfg').read()
config = ConfigParser.RawConfigParser(allow_no_value=True)
config.readfp(io.BytesIO(conf))

# getting data from conf object.
APP_KEY = config.get('twtrauth', 'app_key')
APP_SECRET = config.get('twtrauth', 'app_secret')
OAUTH_TOKEN = config.get('twtrauth', 'oauth_token')
OAUTH_TOKEN_SECRET = config.get('twtrauth', 
                                'oauth_token_secret')

# creating authentication object for twython twitter.
twitter = Twython(APP_KEY, 
                  APP_SECRET, 
                  OAUTH_TOKEN, 
                  OAUTH_TOKEN_SECRET)
\end{python}

\paragraph{Step 2}
\hspace{0pt}\\
Query execution.
\begin{python}
results = twitter.cursor(twitter.search, 
                         q=query, 
                         count="100", 
                         lauage=language)
\end{python}
Where we have the query, and language as input variables to the
\textit{cursor\_extraction} function. 

\paragraph{Step 3}
\hspace{0pt}\\
Data extraction and storage.
\begin{python}
# opens new file with today's date and time now as filename
filename = destination_folder + "/dataset-" + \
    strftime("%d-%b-%Y_%H:%M:%S") 
# opens the file for appending.
data_set = open(filename, 'a')

for result in results:
    # skip previously acquired tweets
    if str(result['id']) in str(previous_tweets_list):
        continue
    else:
        previous_tweets_list.append(result['id'])
    
    # store tweet to file for later use.
    data_set.write(str(result) + "\n")
\end{python}

We open the storage file for writing and gives it a name based on the time of
creation. Then we traverse all results and store tweets we have not seen
before. 

\subsection{Finance}
We get finance data from \url{http://www.netfonds.no/}. Specifically we get the
data about Oslo stock exchange\footnote{OSEBX data:
\url{http://www.netfonds.no/quotes/paperhistory.php?paper=OSEBX.OSE&csv_format=csv}}
We also sort out the data we do not want by breaking at a certain date. 

We do this by using the urllib library in Python: 
\begin{python}
# get data file from internet. (csv)
stock_exchange_history = urllib.urlopen(url).readlines()
records = []
for record in stock_exchange_history:
    # earlier records are not interesting.
    if "2014" in record:
        records.append(record.strip().lower().split(","))
    if "20140414" in record:
        break
return records
\end{python}
%

\section{Dictionary compilation}\label{code:dictionary_compilation}
For the dictionary compilation we use the manually classified datasets to
create mono-, bi-, and trigram dictionaries. 

Starting off the process we run compilation for all datasets: 
\begin{python}
data_files = [
    # [name/description, name of file containing tweets]
    ["kiro", "tweets_classified_manually"],
    ["obama", "obama_tweets_classified_manually"]
]

for item in data_files:
    # get labeled tweets
    # tweets[0] are the positive ones, 
    # tweets[1] are the negative ones.
    tweets = get_tweets_from_manually_labeled_tweets(
             classification_base + item[1])

    compile_monogram_dictionaries(tweets, item[0])
    compile_bigram_dictionaries(tweets, item[0])
    compile_trigram_dictionaries(tweets, item[0])
\end{python}
 
Mono-, bi-, and trigrams creation parts are quite similar: 
\begin{python}
for text in tweets[0]:
    positive_dict.extend(text.split(" "))
# negative
for text in tweets[1]:
    negative_dict.extend(text.split(" "))

filename += "-monogram"
save_dictionaries(positive_dict, negative_dict, filename)
\end{python}

Where the 'positive\_dict =' would be the only part changing.

Bigram:
\begin{python}
bigrams_list = get_bigrams_from_text(text)
positive_dict = positive_dict + bigrams_list
\end{python}

Trigram:
\begin{python}
trigrams_list = get_trigrams_from_text(text)
positive_dict = positive_dict + trigrams_list
\end{python}

We get the bigrams and trigrams by: 
\begin{python}
# if we have no words to work with, return empty list.
text = clean_text(text)
if not text:
    return []

bigrams_list = []
# for all bigram tuples
for tup in bigrams(text.split(' ')):
    # find the tuple texts
    # (u'text1', u'text2')
    m_obj = re.search(r'\(u\'(.+)\', u\'(.+)\'\)', 
                      str(tup).decode())
    if m_obj:
        # combine tuple parts to bigram
        word = m_obj.group(1) + " " + m_obj.group(2)
        # if bigram don't exist
        if word not in bigrams_list:
            # add it to list.
            bigrams_list.append(word)
return bigrams_list
\end{python}
Where we essentially generate sets of words in tuples and then extract the
combined word by string conversion and regex. 

\paragraph{Remove duplicate words}\label{code:remove_duplicate_words}
\hspace{0pt}\\
Then we conveniently remove duplicates between the positive and negative
dictionary before we write the dictionaries to file. 
\begin{python}
primary_dictionary = get_lines_from_file(
    primary_dictionary_name)
secondary_dictionary = get_lines_from_file(
    secondary_dictionary_name)

duplicate_words = []
# for all words in the primary dictionary.
for pw in primary_dictionary:
    #print w
    # for all words in the secondary dictionary.
    for sw in secondary_dictionary:
        # if primary word equals secondary word
        if pw == sw:
            # remove word from both dictionaries.
            primary_dictionary.remove(pw)
            secondary_dictionary.remove(sw)
            duplicate_words.append(pw)
            # as we don't have duplicates within a list,
            # we skip to next pw
            break

# rewrite the updated lists to file
write_array_entries_to_file(primary_dictionary, 
                            primary_dictionary_name)
write_array_entries_to_file(secondary_dictionary, 
                            secondary_dictionary_name)
write_array_entries_to_file(duplicate_words, 
                            "duplicate-words")
\end{python}
%

\section{Sentiment Classification}\label{code:sentiment_classification}
For both classifiers we have utility code for loading data and dictionaries,
but the important parts are the classification. Which is shown here. 

\subsection{Word Count}\label{code:word_count_classification}
For each tweet we do the same thing. Count words and find the difference
between positive and negative words. Here is the code for classifying a tweet.   
\begin{python}
# sanitize text.
tweet = sanitize_tweet(tweet)

# get word count for tweet
word_count = len(tweet.split(' ')) * 1.0

# get word count of pos and neg words.
pos = get_word_count(positive_dict, tweet) / word_count
neg = get_word_count(negative_dict, tweet) / word_count

# storing the polarity value
polarity.append(pos - neg)

if polarity[-1] == threshold:
    edge_cases += 1

# adding sentiment value (True/False), 
# for the last classified tweet.
if polarity[-1] > threshold:
    # positive tweet
    results.append(True)
else:
    # negative tweet
    results.append(False)
\end{python}

\subsection{Using Classifiers}\label{code:classifier_classification}
The execution of the classification can be described in the following steps. 

\paragraph{Classifier initialization}
\hspace{0pt}\\
First we have to initialize the classifier, then train it. We send the class of
the kernel as an input argument, such that \textit{classifier\_class} is
'SklearnClassifier(LinearSVC())'.
\begin{python}
# get the training set.
training_set = nltk.classify.apply_features(
    extract_features_from_text,
    tweets)

# create the classifier.
classifier = classifier_class.train(training_set)
\end{python}

Where the extract\_features\_from\_text function is important. Here the
'dictionary' variable is the combined list of the kiro-monogram-negative.txt and
kiro-monogram-positive.txt dictionaries.
\begin{python}
dictionary = get_list_of_possible_words_in_tweets()
document_words = set(text)
features = {}
# more precision to iterate the word of a tweet then the 
# whole dictionary.
# extracts all words that are both in the dictionary and 
# in the tweet
for word in document_words:
    features['contains(%s)' % word] = (word in dictionary)
return features
\end{python}

\paragraph{Classification}
\hspace{0pt}\\
The code for classifying a tweet is quite simple. 
\begin{python}
# runs classify() on the given classifier class in the 
# nltk library.
results.append(classifier.classify(
    extract_features_from_text(tweet)))
\end{python}

\section{Trend aggregation}\label{code:trend_aggregation}
The trend aggregation is a split process. Where the mining of tweets, sorting,
classification, and trend compilation happens in separate steps. Most important
is the trend compilation. The sorting process only reads tweets
and stores them the file that corresponds with the posting date of the tweet.
Mining is done on the same principles as described earlier, only with other
search words. 

\subsection{Compilation}

First we get positive and negative tweets from the dataset. And store that in a
dict, with the filename, or day, as key. 
\begin{python}
pos, neg = 0, 0

# get tweets for this day.
trend_day_tweets = [ast.literal_eval(tweet) for tweet in
    codecs.open(trend_base + trend_day_filename, 
                'r',
                "utf-8").readlines()]

# sentiment for each tweet.
sentiment = []

# for tweets in this day
for tweet in trend_day_tweets:
    # get words of two or more characters.
    tweet_text = [e.lower() for e in sanitize_tweet(
          tweet['text']).split() if len(e) >= 2]

	# get sentiment
    sentiment.append(classifier.classify(
        extract_features_from_text(tweet_text)))

    # if the last tweet was positive.
    if sentiment[-1]:
        pos += 1
    else:
        neg += 1

return [pos, neg, pos + neg]
\end{python}

Then we transform the tweet data to fit the structure of finance data. 
\begin{python}
def transform_tweet_data():
    #Date,close,high,low,open,volume
    #'trend-Apr-29': {'neg': 21, 'pos': 101, 'tot': 122},
    #print "data,h-l,pos,neg,open,tot"
    trend_days = get_tweet_trend_data()
    keys = sorted(trend_days.keys())
    date = ""
    for i in range(1, len(keys)):
        if "Apr" in keys[i]:
            date = "201404" + str(keys[i].split('-')[2])
        elif "May" in keys[i]:
            date = "201405" + str(keys[i].split('-')[2])
        volume = trend_days[keys[i]]['tot'] * 1.0
        # (positive_tweets / total_amount_of_tweets)*scale
        high = (trend_days[keys[i]]['pos'] / volume) * 1000
        # (negative_teets / total_amount_of_tweets)*scale
        low = (trend_days[keys[i]]['neg'] / volume) * 1000
        openv = 0
        close = high - low
        print str(date) + "," + str(close) + "," + str(high) \
              + str(",") + str(low) + "," + str(volume / 10) \
              + "," + str(0)
\end{python}

\section{Comparison}\label{code:comparison}

For the comparison of trend graphs we have acquired some code.
What we need is the MA and the ADX.

The MA is calculate by: 
\begin{python}
def movingaverage(values, window):
    weigths = np.repeat(1.0, window) / window
    smas = np.convolve(values, weigths, 'valid')
    return smas  # as a numpy array
\end{python}

And for the ADX we have this: 
\begin{python}
def ADX():
    window = 14  # modified by kiro
    PositiveDI, NegativeDI = calcDIs(window)  # modified by kiro

    xxx = 0
    DXs = []

    while xxx < len(date[1:]):
        DX = 100 * ( (abs(PositiveDI[xxx] - NegativeDI[xxx])
                      / (PositiveDI[xxx] + NegativeDI[xxx])))

        DXs.append(DX)
        xxx += 1

    #print len(DXs)
    ADX = ExpMovingAverage(DXs, window)  # modified by kiro
\end{python}

The hundred and some lines used to plot the fancy graphs can be seen in the
code file:
\url{https://github.com/magnuskiro/master/blob/master/code/trend/ma_adx_plotter.py}.

%
 
\section{Issues}\label{code:issues}
The  quality of the code in general is quite good. But whether or not the logic
works out all the time is uncertain. Some of the functions and methods can also
be improved.   

\paragraph{Operating System Impediments}
\hspace{0pt}\\
There might be some OS related specifics that we do not know about. Linux of
the Debian family has been used. So we expect that you know what you are doing
when testing the code on Windows or Mac.

If at all a problem, it will be with pre installed packages that will not
show up in the requirements or library parts discussed earlier. Thats because we
do not know which packages was preinstalled.  

Also some Linux tools has been used in scripts. Those will not work on Windows,
but might on Mac.
