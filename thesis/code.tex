\chapter{The Code}
#TODO write the chapter outline 

#TODO the purpose of the code and the level of completeness. 
Description of the code. It's purpose, and what it does. 

\section{Structure}
There are four main folders. One for classification, one for mining tweets, one
for the dictionaries and on for the files associated with trend. 

Code for classification is split into four files. One file for utilities,
helper functions that does not touch logic. And the three was of classifying
tweets, manual, word counting and with a classifier. List\_threshold\_results
plots the results from the threshold variation in graphs with pyplot. 

The dictionary code is split onto two files, helpers and utility in one, and
logic and execution in the other. 

Trend code is in the trend folder. There we have mining\_utils, which is a
replica of the same file in the twitter folder. This file only helps with the
acquisition of tweets. The trend\_mining file executes the mining operations,
and acquires tweets from twitter based on a set of search terms. The
trend\_tweet\_sorting file sorts tweets from all the raw search term data files
into files based on date. So all tweets from the same date ends in the same
file. And last we have the trend\_compilation file, which contains code for
compiling and plotting trend data and financial data.  

Last we have the 'twitter' directory which has code for extracting tweets from
twitter. This code is used for creating the datasets that the dictionaries are
built from. We have the mining\_utils which is helper code for connection and
writing files and such. While the mining\_operations file does the logic of
managing the acquired tweets. 

The important files are listed under. '- something' are folders, while the
others are actual python files. The indentation shows which files are in which
folders.  
\begin{verbatim}
- code    
    - classification
        classification_utils.py
		list_threshold_results.py
        manual_classification.py
        svm_bayes_classification.py
        word_count_classification.py
    - dictionaries
        dictionaries.py
        dictionary_utils.py
    - trend    
        mining_utils.py
        trend_classification_utils.py
        trend_compilation.py
        trend_mining.py
        trend_tweet_sorting.py
    - twitter
        mining_operations.py
        mining_utils.py
    graph_plots.py
\end{verbatim}

\section{Technology and Libraries}
#TODO write down the technologies and tools used. 
The technology used, frameworks etc.

The used python libraries are: ConfigParser, ast, codecs, io, os, twython,
time, matplotlib, urllib, re, nltk, sklearn.
Most of them are quite standard, while some are not. And some are self
explanatory. 

The libraries that are not shipped with standard python are described in the
following paragraphs. 

\paragraph{Twython}
Library for connection and integration with the twitter api. 
The source and documentation can be found on github:
\url{https://github.com/ryanmcgrath/twython}

\paragraph{nltk}
The natural language tool kit (nltk) is a library that provides functionality
for working with human language. It has functionality such as classifiers and
tokenization tools. See more at \url{http://www.nltk.org/}.

\paragraph{sklearn}
Scikit-learn provides functionality for learning algorithms, machine learning
and classification. We use this library to provide the kernels for out
classification with classifiers. \url{http://scikit-learn.org/}

\paragraph{matplotlib}
Matplotlib is a library for graph plotting. And that is what we use it for.
\url{http://matplotlib.org/}.
 
\section{Data retrieval}
#TODO running of the code and what output to expect. 
\section{Dictionary compilation}
#TODO running of the code and what output to expect. 
\section{Sentiment classification}
#TODO running of the code and what output to expect. 
\section{Trend aggregation}
#TODO running of the code and what output to expect. 
\subsection{Mining}
\subsection{Compilation}
\section{Comparison}
#TODO running of the code and what output to expect. 

\section{Issues}
#TODO code shortcomings and potential improvements. 
Problems in the implementation and the general solution.

#TODO the shortcomings of windows and mac. 

