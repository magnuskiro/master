\chapter{Conclusion}\label{conclusion}

The Twitter API and mining of tweets started as a challenge, but after
working out the quirks, tweets were mined efficiently. The data worked,
but the quality is questionable. The verdict of the data is that it
provide results, and it can be improved.    

Tweets were classified manually and by classifier. Classifiers worked better
than the bag of words method. The classifiers had high accuracy. Which is good.
But also a cause for concern. 

Aggregation of trends worked to some degree. The finance plotting work as
expected, while the Twitter sentiment trend was difficult. The plotted trends
had very few similarities and thus we conclude that technical analysis is still
better than sentiment analysis for predicting trends in the stock market.   

The questions we set out to answer, and the achievements of those, can be
summarised as follows. We can determine the sentiment of a tweet by counting
words, but trained classifiers work better. Trends can be aggregated with data
from Twitter, but the trend indicators created so far is not very useful. We can
say that technical analysis of stock markets work better than sentiment
analysis as a trading tool.
