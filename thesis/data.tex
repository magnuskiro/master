\chapter{Data retrieval and structure}
The data. What used, how and why. Acquired data how and from where. 

System specification and solutions. 

\section{Tweet structure}
There are a lot of meta data in the tweets. In fact most of the data in a tweet
object is not the tweet content itself. 

\section{What data was used}
\section{How to Obtain the tweets}
Simple guide to access the twitter api:  http://datascienceandprogramming.wordpress.com/2013/05/14/twitter-api/

\section{Data structure}
\section{Problems and shortcomings}

\section{Dictionaries}\label{sec:dict}

Tim Loughran and Bill McDonald has a set of dictionaries available from the
websites of University of Notre Dame \footnote{fiks tekst: nd.edu:
\url{http://www3.nd.edu/~mcdonald/Word_Lists.html}}. 

List of Dictionaries:
\begin{itemize}
    \item negative words
General list of negative words. No particular category. Used for basic   
    \item positive words
This dictionary contains a small set of positive words. There are no general
category for the words. The words are not directly related to finance. 
    \item Uncertainty words
    \item litigious words
    \item modal words strong
    \item modal words weak
\end{itemize}

%Dictionary expansion. 
There are a lot of words that are not classified yet. Those words should be
stored and classified later. This to improve the classifiers potential to
correctly classify tweets.  

