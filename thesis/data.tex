\chapter{Data, retrieval and structure}
The data. What used, how and why. Acquired data how and from where. 

System specification and solutions. 

This section describes the data sources used in this thesis. 

We describe the methods for acquiring data in each source. 

Further, the structure, meta data, and characteristics of the data is
described. 

% data sources
\section{The Sources}
\paragraph Twitter 
Twitter is the source of the base data for sentiment analysis.

\paragraph Dictionaries
The list of word used in the classification of tweets. 

\paragraph <Finance Source>
While <Finance Source> is where we get the stock exchange data which is used in
comparison to the tweet trend TODO:ref -\ref{ref to tweet trend.}. 


% The section about twitter. 
\section{Twitter}

Please note that Twitter's search service and, by extension, the Search API is
not meant to be an exhaustive source of Tweets. Not all Tweets will be indexed
or made available via the search interface. 

% data acquisition
\subsection{Twitter API}
Description of the api and which options we have with the search. 

Simple guide to access the twitter api:  http://datascienceandprogramming.wordpress.com/2013/05/14/twitter-api/

This is a list of all the API calls that is used in this thesis. 
\begin{itemize}
\item[q:] A UTF-8, URL-encoded search query of 1,000 characters maximum, including
operators. Queries may additionally be limited by complexity.

\end{itemize}

\subsection{Mining tweets} % data retrieval.
%Simple guide to access the twitter api:  http://datascienceandprogramming.wordpress.com/2013/05/14/twitter-api/

Specifics about the api. 

Volume

Time constraints 

Query constraints. 

\subsection{Dataset(s)}
We are aiming to use multiple sets of 10k tweets. 

This many tweets in each set. etc. 

Three sets based on three different search terms.
Split the original three sets into 10 subsets. 
Take one subset from each superset and manually classify them.Then
automatically classify the 27 other sets.  

\paragraph{10k-tweets-base}
The first dataset.
Based on tweets with the search term "Finance". 
The interesting part is that 5206 of 10.000 tweets contain the word "finance".

Usage. Splitting this dataset into 10 parts, so that I can use each set for a
different version of the classification. 

\subsection{Tweet structure}
There are a lot of meta data in the tweets. In fact most of the data in a tweet
object is not the tweet text itself.  

Tweets directly usable in python 

\subsection{Biased data}
Due to the necessity of a search term in the query, we only get tweets that are
related to the 

\subsection{Problems and shortcomings}
The potential problems and shortcomings of the data. 

% describing the dictionaries used in the classification of tweets. 
\section{Dictionaries}\label{sec:dict}

Tim Loughran and Bill McDonald has a set of dictionaries available from the
websites of University of Notre Dame \footnote{fiks tekst: nd.edu:
\url{http://www3.nd.edu/~mcdonald/Word_Lists.html}}. 

List of Dictionaries:
\begin{itemize}
    \item negative words
General list of negative words. No particular category. Used for basic   
    \item positive words
This dictionary contains a small set of positive words. There are no general
category for the words. The words are not directly related to finance. 
    \item Uncertainty words
    \item litigious words
    \item modal words strong
    \item modal words weak
\end{itemize}

%Dictionary expansion. 
There are a lot of words that are not classified yet. Those words should be
stored and classified later. This to improve the classifiers potential to
correctly classify tweets.  

\subsection{Expanded dictionaries}
When classifying a tweet we get a number of words that does not have a previous
classification. These words are classified and added to the expanded
dictionaries to form better results. 


% The financial data used in the thesis. 
\section{Financial source}
* obtaining the data(potential mining operations)
* about the dataset  
* structure
* potential problems 
