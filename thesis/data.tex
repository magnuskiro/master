\chapter{Data, retrieval and structure}
% The data. What used, how and why. Acquired data how and from where. 

This section describes the data sources, methods for acquisition, and the
structure of the used data. \ref{data:tweets} describes twitter and the mined
tweets. \ref{data:dictionaries} describe the different lists of words used in
the classification process. And last the finance data is described in
\ref{data:finance}.

For each section the structure, characteristics, metadata and usage are
described. 
%

% The section about twitter and tweets 
\section{Tweets}\label{data:tweets}
#TODO introduce and describe a tweet. 
Please note that Twitter's search service and, by extension, the Search API is
not meant to be an exhaustive source of Tweets. Not all Tweets will be indexed
or made available via the search interface. 
%

\subsection{Tweet Structure}
#TODO describe the json structure and the contents.
There are a lot of meta data in the tweets. In fact most of the data in a tweet
object is not the tweet text itself.

Tweets directly usable in python, vie dictionary and literal eval thingy. 
%

% data acquisition
\subsection{Twitter API}
#TODO describe the API
#TODO api setup
#TODO API simple use.
#TODO API restrictions

Description of the api and which options we have with the search. 

Simple guide to access the twitter api:  http://datascienceandprogramming.wordpress.com/2013/05/14/twitter-api/

This is a list of all the API calls that is used in this thesis.
% 

\paragraph{Search} 
\begin{itemize}
\item[q] A UTF-8, URL-encoded search query of 1,000 characters maximum, including
operators. Queries may additionally be limited by complexity.

\item[count] The amount of tweets acquired in each request. Standard = 15, max
= 100. 

\end{itemize}
%

\paragraph{Mining optimization}
#TODO -rt, searching vs generator 
%

\subsection{Tweet sets}
#TODO manual classification
#TODO search terms
#TODO limitations

We are aiming to use multiple sets of 10k tweets. 

This many tweets in each set. etc. 

Three sets based on three different search terms.
Split the original three sets into 10 subsets. 
Take one subset from each superset and manually classify them.Then
automatically classify the 27 other sets.  

\subsection{Biased Data}
#TODO write this section
Due to the necessity of a search term in the query, we only get tweets that are
related to the given terms.

Further more the datasets of manually labeled tweets are biased based on my
personal opinion and state of mind in the moment of classification.  

\subsection{Trend Data}
#TODO Briefly describe the mining and API shortcomings for this particular use.
#TODO describe the trend search terms: '_search-terms'
#TODO write shortcomings of the search terms. 
#TODO Describe the tweet data sets and sorting.  

\subsection{Problems, Shortcomings, and Possible Improvements}
The potential problems and shortcomings of the data. 

#TODO retweets. 
#TODO search terms.
#TODO finance vs not finance.
#  

% describing the dictionaries used in the classification of tweets. 
\section{Dictionaries}\label{data:dictionaries}
#TODO introduction to dictionaries, of corpus whatever the name. 
#TODO the purpose of the dictionary
#TODO use of the dictionaries. 
%

\subsection{Downloaded Dictionaries}
#TODO as an example of dictionaries not compiled. 

\paragraph{Obama}
#TODO describe obama dictionary. 
\paragraph{Loughran & McDonald}
#TODO describe this dictionary
Tim Loughran and Bill McDonald has a set of dictionaries available from the
websites of University of Notre Dame \footnote{#TODO fiks tekst: nd.edu:
\url{http://www3.nd.edu/~mcdonald/Word_Lists.html}}. 

List of Dictionaries:
\begin{itemize}
    \item negative words
General list of negative words. No particular category. Used for basic   
    \item positive words
This dictionary contains a small set of positive words. There are no general
category for the words. The words are not directly related to finance. 
    \item Uncertainty words
    \item litigious words
    \item modal words strong
    \item modal words weak
\end{itemize}
%

\subsection{Compiled Dictionaries}

#TODO describe the dictionary compilation.
#TODO describe the different dictionaries 
#TODO list all dictionaries with plot graph of threshold. 

Monogram, obama
1 and 10

Monogram LoughranMcDonald
2 and 11

Monogram, combined Obama and LoughranMcDonald
3 and 12

Kiro, Monogram, self compiled
4 and 13

Obama, Monogram, self compiled
5 and 14

Kiro, Bigram, self compiled
6 and 15

Obama Bigram, self compiled
7 and 16

Kiro, Trigram, self compiled
8 and 17

Obama Trigram, self compiled
9 and 18
%

\subsection{Error analysis og duplicate words removal}
#TODO write about it and analyse:
'duplicate-words-from-monogram-compilation.txt'
Most stop words and other insignificant words are removed. 
%

% The financial data used in the thesis. 
\section{Finance Data}\label{data:finance}
#TODO obtaining the data(potential mining operations)
#TODO about the dataset, csv
#TODO potential problems 
%

