\chapter{Future Work}\label{future_work}
Through the time spend writing this thesis a lot of ideas and edging areas of
research have been touched, albeit many of them only briefly and only with
little interest. This chapter look into all the things not
quite touching the core of this thesis.
%

\section{Twitter}\label{future_work:twitter}
Twitter should be used for further mining, specific areas of interest, and
specific users and hashtags should be investigated. 
The finance section should also be an area of focus. 
The importance of retweets should also be investigated. How does retweets
propagate a tweet through social media and what impact does it have?

\paragraph{Twitter API}
\hspace{0pt}\\
We should look at the different endpoints of the API and see how we can
use them smartly \footnote{Twitter: \url{https://dev.twitter.com/docs/api/1.1}}.
There are definitely unused options that can improve the sentiment analysis, or
data mining aspects discussed in this thesis. 

\paragraph{Tweet sets}
\hspace{0pt}\\
There should have been a lot more data and it should have been more diverse. To
achieve this the natural path would be to define a set of finance words that we
define as the finance area of twitter. Then we should use those words to mine
Twitter for finance relevant tweets. The tweets would be used to create better
dictionaries and provide better sentiment data in trend aggregation. 

\paragraph{Special Twitter content}
\hspace{0pt}\\
Twitter has it's own convention of symbols that is used. Symbols like \#hashtags,
@usernames and stock markers, \$STO. What is the contribution of those? Do these
special symbols add special meaning to a tweet? Questions like these and other
questions about use and frequency should be answered. We think that this can
contribute to the sentiment classification and trend aggregation. 

\paragraph{Language analysis}
\hspace{0pt}\\
Some form of language analysis should be done with tweets to see if the
language used on twitter can be described in a good systematic way. Also to
find specific words, sentences, punctuation and other syntactic sugar. 
As an example of this we have 'bc'. It is an abbreviation that in it's
context means 'because'. The existing information from SMS
linguistics\footnote{Wikipedia:
\url{https://en.wikipedia.org/wiki/Texting_Slang}} should be
tested on Twitter to see it's relevance. 
%

\section{Dictionaries}\label{future_work:dictionaries}
For the dictionaries and the compilation of them we should look further into
the quality of them. Improvements with stop words and further elimination of
neutral words is also an option. This could be some kind of qualitative
analysis of words and their sentiment. Combinations of mono, bi and trigrams
should also be tested. Do we get better results if we extract features that are
mono, bi and trigram? 
%

\section{Sentiment}\label{future_work:sentiment}
The future of sentiment classification lies with the classifiers. But the
feature extraction can be improved. This is a natural place to start further
research into sentiment analysis. A further test of training data and test sets
should be done to validate the accuracy of the classifiers.   

We should cross classify and test the classification with the SVM classifier
and do quality assurance on the work with it. We should also look at other
classifiers and other data to confirm the findings we have so far.
 
\paragraph{Is neutral negative or positive?}
\hspace{0pt}\\
If a tweet is neutral, where the amount of positive words and
negative words are the same, how do we then classify the tweet? The edge case
where we, with the threshold, found that many tweets were classified wrongly.
How can this problem be improved?

\paragraph{Word weighting}
\hspace{0pt}\\
How can we use weighting of words in sentiment classification? Are there a way
where we can assign a value to some specific words to improve the bag of words
classification? Is 'good' as positive as 'excellent'? Or 'bad' as negative as
'evil'? 

\paragraph{How people express sentiment}
\hspace{0pt}\\
We should explore the psychology of perception and classification. How do
people perceive content differently? Is finance easier to label than politics?
A bit of a psychological twist is to look into the natural language of people,
and find out how we express a sentiment. What trigger an event and what part
of the event is significant to the sentiment, and can knowledge about how
people express sentiment improve upon the classification of sentiment? 

\paragraph{Use of sentiment analysis on news}
\hspace{0pt}\\
As a test we can change the dataset and focus on news. Particularly finance
related news, and focus on the title and ingress of the news. In many cases the
sentiment and all the necessary information is presented in the title and
ingress. The title and ingress can in many ways be compared to tweets, short
messages containing a sentiment.   
%

\section{Trend}\label{future_work:trend}
\paragraph{Credibility}
\hspace{0pt}\\
The credibility of a tweeter might have an impact on the sentiment. We do not
know. Reputation and social reach are two factors that cannot be ignored when 
classifying tweets. Do these factors have an effect on the trend at all? These
factors should be considered when aggregating a trend. 

\paragraph{Trend Data}
\hspace{0pt}\\
The data we have acquired should be further analysed. Is the data good? Is the
data representative for twitter? Can we use the data in other ways? Is the
quality of the data any good?  The analysis of the trend data is important for
further improvements for the trend aggregation. We have over 30k of tweets
waiting to be analysed. 

\paragraph{Trend Aggregation}
\hspace{0pt}\\
We have to look for other methods of aggregating a trend. This to improve the
Twitter trend we have created, but also to validate the accuracy and
correctness of the trend we have created. Other parts of the tweet data might
be used also. This should be explored.  
%
