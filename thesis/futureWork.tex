\chapter{Future Work}\label{future_work}
Through the time spend writing this thesis a lot of ideas and edging areas of
research have touched. Many of them only briefly and only with little interest.
This is the chapter we look further into all the things not quite touching the
core of this thesis.   

For the parts of future work we have Twitter, \ref{future_work:twitter},
Dictionaries, \ref{future_work:dictionaries}, Sentiment,
\ref{future_work:sentiment}, and Trend, \ref{future_work:trend}.
%

\section{Twitter}\label{future_work:twitter}
Mainly Twitter should be used further to mine specific areas of interest. Dig
deeper into the finance section or focus on another category of knowledge.

\paragraph{Twitter API}
We should look at the different endpoints of the API and see how see how we can
use them smartly \footnote{Twitter: \url{https://dev.twitter.com/docs/api/1.1}}.
There are definitely unused options that can improve the sentiment analysis, or
data mining aspects discussed in this thesis. 

\paragraph{Tweet sets}
One of the critiques that can be said about this thesis is the data used. It
should have been a lot more of it and it should have been more diverse. To
achieve this the natural path would be to define a set of finance words that we
define as the finance area of twitter. Then use those words to mine tweets from
twitter. The tweet sets should be expanded to use a wide range of finance words.
Then create dictionaries from the sets. 

\paragraph{Special Twitter content}
Twitter has it's own convention of symbols that is used. Symbols like \#hashtags,
@usernames and stock markers, \$STO. What is the contribution of those? Do these
special symbols add special meaning to a tweet? Questions like these and other
questions about use and frequency should be answered. 

\paragraph{Language analysis}
Some form or language analysis should be done with tweets. To see if the
language used on twitter can be described in a good systematic way. And also to
find specific words, sentences, punctuation and other syntactic sugar. 
As an example of things to look for we have abbreviations such as 'bc'. 'bc'
means because in the setting the word was found. The existing information from
SMS linguistics should be tested on Twitter to see it's relevance.
%

\section{Dictionaries}\label{future_work:dictionaries}
For the dictionaries and the compilation of them we should look further into
the quality of them. Improvements with stop words and further elimination of
neutral words is also an option. This could be some kind of qualitative
analysis of words and their sentiment. Combinations of mono, bi and trigrams
should also be tested. Do we get better results if we extract features that are
mono, bi and trigrams? 
%

\section{Sentiment}\label{future_work:sentiment}
The future of sentiment classification lies with the classifiers. But the
feature extraction can be improved. This is a natural place to start further
research into sentiment analysis. A further test of training data and test sets
should be done to validate the accuracy of the classifiers.   
 
\paragraph{Is neutral negative or positive?}
If a tweet is neutral, where the amount of positive words and
negative words are the same, how do we then classify the tweet? The edge case
where we, with the threshold, found that many tweets were classified wrongly.
How can this problem be improved?

\paragraph{Word weighting}
How can we use weighting of words in sentiment classification? Are there a way
where we can assign a value to some specific words to improve the bag of words
classification? Is 'good' as positive as 'excellent'? Or 'bad' as negative as
'evil'? 

\paragraph{How people express sentiment}
A bit of a psychological twist is to look into the natural language of people,
and find out how we express a sentiment. What triggers the event and what part
of the event is significant to the sentiment? And can knowledge about how
people express sentiment improve upon the classification of sentiment? 

\paragraph{Use of sentiment analysis on news}
As a test we should change the dataset and focus on news. Particular finance
news. And only use the title and ingress of the news. In many cases the
sentiment and all the necessary information is presented in the title and
ingress. The title and ingress can in many ways be compared to tweets. Short
messages containing a sentiment.   
%

\section{Trend}\label{future_work:trend}
\paragraph{Credibility}
The credibility of a tweeter might have an impact on the sentiment. We do not
know. Reputation and social reach are two factors that cannot be ignored when 
classifying tweets. These factors should be considered when aggregating a
trend. Do the factors have an effect on the trend at all? 

\paragraph{Trend Data}
The data we have acquired should be further analysed. Is the data good? Is the
data representative for twitter? Can we use the data in other ways? Is the
quality of the data any good?  The analysis of the trend data is important for
further improvements for the trend aggregation. We have over 30k of tweets
waiting to be analysed. 

\paragraph{Trend Aggregation}
We have to look for other methods of aggregating a trend. This to improve the
Twitter trend we have created, but also to validate the accuracy and
correctness of the trend we have created. Other parts of the tweet data might
be used also. This should be explored.  
%
