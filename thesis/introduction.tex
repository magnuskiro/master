\chapter{Introduction}

\section{What}
What has been done and what was going to happen.
What is this thesis about? What are we doing? What are the goals of this
thesis? What is the setting for this thesis, the circumstances and environment
of the work. 

\section{Why}
Why we do this and the motivation we have for doing this.
Why is this work done? Why do we benefit from this? Why do I want to do this?
Why is this relevant for others? 

\section{Research questions}
	\paragraph{Can tweets be used to determine a trend in the stock market?\\}
	We will look at the usefulness of tweets as a way to extract sentiment. And
then look at possible applications for the sentiment in the stock market. 
Does the use of positive and negative words in the classification of tweets
have an impact on the determination of the trend and what effects does the
polarity of a tweet have on the trend?

	\paragraph{How can twitter be used to predict trends in the stock market?\\}
* On a more abstract level we would like to find out whether or not twitter as a
microblogging site can be used to predict trends in the stock market.

* We would also like to find out which parts of twitter are the most useful
ones. 

* Which twitter sources are most suitable for predicting the stock market
trend?

* Credibility, what sort of credibility level has to be attained to certify the
quality of the trend prediction. 

	\paragraph{How does these trends compare to the technical analysis in the
stock market?\\}
	The comparison of compiled trends and the moving average of the stock
market will  give us insight in to the possibility to predict financial trends
over longer periods of time then a few days. Researchers has already looked at
the possibility to predict the direction of the market tomorrow based on the
volume of tweets of today an their sentiment. 

\section{Findings}

\section{Outline}
The outline of the document and the description of what which part is about. 


