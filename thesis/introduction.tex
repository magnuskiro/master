\chapter{Introduction}\label{introduction}

\section{What}
#TODO write what this thesis contains and what is the goal of the thesis. 

What is this thesis about? 

What are we doing? 

What are the goals of this thesis? 

What is the setting for this thesis, the circumstances and environment
of the work. 

\section{Why, Motivation}
#TODO write why I want to do this and why we want to look at these specific
points. 

Why we do this and the motivation we have for doing this.

Why is this work done? Why do we benefit from this? Why do I want to do this?

Why is this relevant for others? 

\section{Research questions}
% How can we classify a tweet sentiment. 
\subsection{How do we determine the sentiment of a tweet?\\}
Can we extract knowledge from tweets to find a sentiment?
	
We will look at the usefulness of tweets as a way to extract sentiment. 

Which parts of a tweet is useful for the classification of a tweets sentiment?

Which methods are best to classify tweets? 

How do we best find the sentiment of tweets?

% Can we create a trend from data on twitter.
\subsection{How can twitter be used to aggregate a trend?\\}
Can we build a trend based on information from tweets? 
 
Can Twitter as a microblogging site be used as a data source in aggregation of trends.

Credibility, what sort of credibility level has to be attained to certify the
quality of the trend prediction. 

Which parts of twitter are most useful to generate a trend?

% compare a twitter trend with a financial one. 
\subsection{How does trends on twitter compare to technical analysis in the
stock market?\\}
Technical analysis compared with the tweet trend.

We will look at possible applications for the sentiment in the stock market.

Which twitter sources are most suitable for predicting the stock market
trend?

In finance, the moving average is a result of technical analysis. This and
other trend defining qualities of financial data is used to compile trends. 

Twitter has data such as the amount of tweets posted today, the location where
tweets are posted from, and which users has posted. Aggregated, these data
become represents a trend.  

Previously researchers have managed to predict direction of the market the
next few days based on the volume of tweets. 

We are interested in the correlation between trends on twitter and the moving
average in finance. Hopefully this will give some insight of how the sentiment on
Twitter influences the sock market.  

\section{Findings}
We explore the usage of lists of words, dictionaries, in sentiment analysis.
And we look at data retrieval from twitter and the trend we can create from it. 
With varying results we get positive results with the dictionaries, while the
trend aggregation lacks the finesse and results it should have had. 

\section{Outline}
The nine chapters of this thesis all describe different aspects of the work
done. First we have this introduction, \ref{introduction}. Naturally we follow
with previous word, \ref{previous_work}, the state of the art part, where we
discuss the newest research in this field. The data retrieval is next in
chapter \ref{data}. 

Proceeding the data chapter we go for the core of the thesis, the sentiment
analysis, \ref{sentiment}. The sentiment naturally leads up to the trend in
chapter \ref{trend}. Code, \ref{code}, comes next to explain how we did things
and go deeper in the specifics of implementation. Results and discussion comes
in chapter \ref{results}. While conclusion, \ref{conclusion}, and future work,
\ref{future_work}, rounds off the thesis.
%
