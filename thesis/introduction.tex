\chapter{Introduction}\label{introduction}

\section{Context}
This thesis have largely three parts to it. The finance part, where we look at
stock market prediction\footnote{Wikipedia:
\url{http://en.wikipedia.org/wiki/stock_market_prediction}}, and technical
analysis.  

It is well known that people are affected by sentiment. Sentiment in finance is
not new concept, but it is has not been explored very much yet. So we want to
see if there are ways we can predict trends with the help of sentiment. We
generate trends based on sentiment and stock exchange data. And we will look for
correlations between them. 

For the sentiment we need data. Twitter is a great source of data and a better
place to express sentiment. So we chose twitter as our primary source of
sentiment. To use the data in some smart ways we compile dictionaries from
tweets. And use the dictionaries for sentiment classification. 

The elements that are the most difficult are the creation a trend based on
tweets, and getting representative data. 
%

\section{Motivation}
We have an interest to find out how sentiment can be extracted from twitter and
how it effects finance trends. Further we are motivated by the social media
aspect of trends and how that effects finance. 

Why is this work done? Why do we benefit from this? Why do I want to do this?
We believe that microblogging and social media plays a role in todays trends
and we would like to explore that. Twitter as a social media platform is a huge
place for companies to post updates and provide customer support, so it is a
natural place to gather data for analysis. 

If everything works out and we strike gold, we would hopefully find relations
between twitter content and financial trends. And use that to prove increase
revenue in trade. Our work is relevant for further endeavors about twitter,
sentiment and it's relation to finance. 

\section{Research questions}\label{introduction:research_questions}
#TODO specify a bit better.\\ 

% How can we classify a tweet sentiment. 
\subsection{How do we determine the sentiment of a
tweet?}\label{introduction:rq1}
Can we extract knowledge from tweets to find a sentiment?
	
We will look at the usefulness of tweets as a way to extract sentiment. 

Which parts of a tweet is useful for the classification of a tweets sentiment?

Which methods are best to classify tweets? 

How do we best find the sentiment of tweets?

% Can we create a trend from data on twitter.
\subsection{How can twitter be used to aggregate a
trend?}\label{introduction:rq2}
Can we build a trend based on information from tweets? 
 
Can Twitter as a microblogging site be used as a data source in aggregation of trends.

Credibility, what sort of credibility level has to be attained to certify the
quality of the trend prediction. 

Which parts of twitter are most useful to generate a trend?

% compare a twitter trend with a financial one. 
\subsection{How does trends on twitter compare to technical analysis in the
stock market?}\label{introduction:rq3}
Technical analysis compared with the tweet trend.

We will look at possible applications for the sentiment in the stock market.

Which twitter sources are most suitable for predicting the stock market
trend?

In finance, the moving average is a technique in technical analysis. This and
other trend defining qualities of financial data is used to compile trends. 

Twitter has data such as the amount of tweets posted today, the location where
tweets are posted from, and which users has posted. Aggregated, these data
become represents a trend.  

Previously researchers have managed to predict direction of the market the
next few days based on the volume of tweets. 

\section{Methods}

For the classification of sentiment we will use bag of words\footnote{Wikipedia:
\url{https://en.wikipedia.org/wiki/Bag_of_words_model}}, SVM, and Naive Bayes.
SVM and Naive Bayes to test if the classifiers perform better than bag of words
with dictionaries in classification. Bag of words is mostly used to test
dictionaries.

Moving average\footnote{Wikipedia:
\url{https://en.wikipedia.org/wiki/Moving_average}}, and average directional
index\footnote{StockCharts.com
\url{http://stockcharts.com/school/doku.php?id=chart_school:technical_indicators:average_directional_index_adx}}
are used to plot trend graphs. The trend graphs can be seen in section
\ref{experiments:trend}, on
page \pageref{experiments:trend}. Concepts and explanations of how the trend
works can be found in section \ref{trend}, on page \pageref{trend}

\section{Findings}
We explore the usage of lists of words, dictionaries, in sentiment analysis.
And we look at data retrieval from twitter and the trend we can create from it. 
With varying results we get positive results with the dictionaries, while the
trend aggregation lacks the finesse and results it should have had. 

\section{Outline}
The nine chapters of this thesis all describe different aspects of the work
done. First we have this introduction, \ref{introduction}. Naturally we follow
with previous word, \ref{previous_work}, the state of the art part, where we
discuss the newest research in this field. The data retrieval is next in
chapter \ref{data}. 

Proceeding the data chapter we go for the core of the thesis, the sentiment
analysis, \ref{sentiment}. The sentiment naturally leads up to the trend in
chapter \ref{trend}. Code, \ref{code}, comes next to explain how we did things
and go deeper in the specifics of implementation. Results and discussion comes
in chapter \ref{results}. While conclusion, \ref{conclusion}, and future work,
\ref{future_work}, rounds off the thesis.
%
