\chapter{Introduction}\label{introduction}

\section{Context}
This thesis consist of three parts. 

First the Twitter aspect. Where we use Twitter as a data source for sentiment
analysis. Second, the sentiment analysis in itself. Creation of dictionaries
and how we determine the sentiment of a tweet. Third, the finance part, where we
look at stock market prediction\footnote{Wikipedia:
\url{http://en.wikipedia.org/wiki/stock_market_prediction}}, and technical
analysis. 

It is well known that people are affected by sentiment. Sentiment analysis in
finance is not a new concept, but it is has not been explored very much yet.
We want to see if there are ways we can predict trends with the help of
sentiment analysis. We generate trends based on sentiment, and stock exchange
data.  

Twitter is a great source of data, and a
good place to express sentiment. So we chose Twitter as our primary source of
sentiment. To use the data in some smart ways we compile dictionaries from
tweets. And use dictionaries for sentiment classification. 

Our greatest problem is creating a sentiment trend graph based on data from
tweets. 
%

\section{Motivation}
We want to find how sentiment can be extracted from Twitter and
how it effects finance trends. We are motivated by the fact that social media
has become a bi part of peoples lives. 

Why is this work done? We believe that microblogging and social media are great
places to find and observe trends, and we would like to explore that. Twitter as
a social media platform is a huge place for companies to post updates and
provide customer support, so it is a natural place to gather data for sentiment
analysis.

We would hopefully find relations between Twitter content and financial trends,
and use that to prove increase revenue in trade. Our work is relevant for
further endeavors about Twitter, sentiment and it's relation to finance.  

\section{Research questions}\label{introduction:research_questions}
% How can we classify a tweet sentiment. 
\subsection{How can we determine the sentiment of a
tweet?}\label{introduction:rq1}

The factors we look at are how, if possible, can we extract knowledge from
tweets and then use that to find the sentiment of said tweet. The usefulness of
a tweet as a source for sentiment is in question. What parts of a tweet are
useful in sentiment analysis? Which methods can we use to classify a tweet, and
which method is best?

% Can we create a trend from data on Twitter.
\subsection{How can Twitter be used to aggregate trends?}\label{introduction:rq2}
As trends are interesting and useful we aim to create a trend based on data
from Twitter. We ask ourselves if Twitter as a microblogging site can be used as a
data source in aggregation of trends. The credibility of a tweet is also an
aspect we would like to investigate. To create a trend we need to find which
parts of Twitter can be used. Combining these, hopefully, we can create trends. 

% compare a Twitter trend with a financial one. 
\subsection{How does trends from Twitter compare to technical analysis in the
stock market?}\label{introduction:rq3}
In finance, the moving average is a method in technical analysis to predict
trends. Can methods of technical analysis or other trend defining qualities of
financial data be used to predict trends? This and other trend methods of
technical analysis data is used to predict trends.  We will look at possible
applications for sentiment in the stock market.

Twitter has data such as the amount of tweets posted today, the location where
tweets are posted from, and which users has posted. Which Twitter sources are
most suitable for predicting the stock market trend? Aggregated, can these data
represent a trend? Previously researchers have managed to predict
direction of the market the next few days based on the volume of
tweets\cite[]{doukas10:sentiment_and_momentum}.  

Ultimately we will compare trends in technical analysis with trends based on
sentiment from twitter. 

\section{Methods}

In this thesis we have two main ways of classifying the sentiment of tweets.
Word counting and training a classifier. The word counting is the known method
'bag of words'
\footnote{Wikipedia: \url{https://en.wikipedia.org/wiki/Bag_of_words_model}},
and the training of classifiers use support vector machine, and Naive Bayes.
The trained classifiers provide insight and comparison of which classifying
method is best.  

Moving average\footnote{Wikipedia:
\url{https://en.wikipedia.org/wiki/Moving_average}} and average directional
index\footnote{StockCharts.com
\url{http://stockcharts.com/school/doku.php?id=chart_school:technical_indicators:average_directional_index_adx}}
are used to plot trend graphs in finance.
Based on the moving average and the average directional index we create our own
adaption to create sentiment trends.  
The trend graphs can be seen in section
\ref{experiments:trend}, on page \pageref{experiments:trend}. Concepts and
explanations of how the trend works can be found in section \ref{trend}, on page \pageref{trend}

\section{Outline}
The nine chapters of this thesis all describe different aspects of the work
done. 
Introduction, chapter \ref{introduction}, introduces the context and describes
what we aim to do and out goals. 

Background and Previous work, chapter \ref{previous_work}, presents other
research in the same area and relevant background knowledge. 

Data, retrieval and structure, chapter \ref{data}, describes the data sources,
how we use them, and the structure of the data we use. 

Sentiment Classification, chapter \ref{sentiment}, describes the methods we use
to determine sentiment of tweets.  

Trending, chapter \ref{trend}, considers the different aspects of a trend, on
twitter and in finance. How trends are aggregated and how they are used is also
covered.  

Experiments, chapter \ref{experiments}, contains the specifics of how we used
our methods, and how we test the research questions.

Results and Discussion, chapter \ref{results}, presents what we find with our
experiments. And have a discussion about the results we get.   

Conclusion, chapter \ref{conclusion}, concludes this thesis and highlights the
findings.  

Future Work, chapter \ref{future_work}, covers the next logical steps in
sentiment analysis with twitter and trends.  

Also noteworthy is appendix \ref{code}, where we detailed describe what the
code does.  
%
