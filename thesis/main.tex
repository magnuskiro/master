% latex article template

% cheat sheet(eng): http://www.pvv.ntnu.no/~walle/latex/dokumentasjon/latexsheet.pdf
% cheat sheet2(eng): http://www.pvv.ntnu.no/~walle/latex/dokumentasjon/LaTeX-cheat-sheet.pdf
% reference manual(eng): http://ctan.uib.no/info/latex2e-help-texinfo/latex2e.html

% The document class defines the type of document. Presentation, article, letter, etc. 
\documentclass[12pt, a4paper]{article}

% packages to be used. needed to use images and such things. 
\usepackage[pdfborder=0 0 0]{hyperref}
\usepackage[utf8]{inputenc}
\usepackage[english]{babel}
\usepackage{graphicx}
\PassOptionsToPackage{hyphens}{url}

% hides the section numbering. 
\setcounter{secnumdepth}{-1}

% Graphics/image lications and extensions. 
\DeclareGraphicsExtensions{.pdf, .png, .jpg, .jpeg}
\graphicspath{{./images/}}

% Title or header for the document. 
\title{
	Sentiment analysis of Tweets in correlation with financial investments. 
}
% Author
\author{
	Magnus L Kirø \\
	Student, IDI, IME, NTNU \\
	\\
	CO authors here.
}
\date{\today}

\begin{document}
\maketitle
\pagenumbering{arabic}

\begin{abstract}
Twitter har become mainstream and it's significance in the financial world is
increasing. This makes tweets interesting as a source of stock predictions.

This article will look at the state of the art techniques that exist and
compare them. Keeping in mind the trends that crystalises from the underlying
sentiment analysis of stock exchange relevant tweets. 

What is state of the art in sentiment analysis with twitter as the datasource? 

A summary of relevant research will be provided. And thoughts of missing
aspects of the current results will uttered. 

\end{abstract}

\section{Introduction}
Log test

\paragraph{Outline}
The remainder of this article is organized as follows.
Section~\ref{previous work} gives account of previous work.
Our new and exciting results are described in Section~\ref{results}.
Finally, Section~\ref{conclusions} gives the conclusions.

\section{Previous work}\label{previous work}
A much longer \LaTeXe{} example was written by Gil~\cite{Gil:02}.

\section{Results}\label{results}
In this section we describe the results.

\section{Conclusions}\label{conclusions}
We worked hard, and achieved very little.

% The following has nothing to do with structure or content. It's just nice to have easily available without googling. 
\section{Useful stuffs}

% basic table
% http://en.wikibooks.org/wiki/LaTeX/Tables
% the "{ l c r }" part decides if the content of a cell should be at the center, left or right. 
\begin{tabular}{ l c r }
  1 & 2 & 3 \\
  4 & 5 & 6 \\
  7 & 8 & 9 \\
\end{tabular}

$_This is subscript$
$^This is Superscript$

% imgae example. 
\begin{figure}[htb]
    \centering
    \includegraphics[width=\textwidth]{nameOfImageFile} 
    \caption{The text that shows under the image, image text.}
    \label{fig:FigureLableName}
\end{figure}


\bibliographystyle{abbrv}
\bibliography{main}

\end{document}
This is never printed
