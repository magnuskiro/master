
\chapter{Sentiment Analysis}
This section describes the experiments done. High level description and
execution of experiments. Detailed execution and technical details in appendix. 

\section{Why use sentiment analysis?}
\section{Tweet sentiment}
\subsection{Techniques}
\section{Sentiment trend}
\section{Experiments}
% todo figure out what experiments are smart to do. 

* Experiment with the time frame of the prediction of the trend. 
	* Typically the variation of time. What's the longest into the future that
we can predict the trend?
* 

\subsection{Sentiment classification with lexicons of positive and
negative words.}
Common experiment to acquire knowledge of the field, and see if the given
dictionaries are usable. 

A basic approach to classification is to count words. Positive and negative
words. To do this it is necessary have dictionaries to provide the
classification of simple words. Section:\ref{sec:dict} describes the
dictionaries used. Some of them are quite simple, while others are more
extensive. 

The polarity of a given tweet is based on the difference in the amount of
positive verses negative words. 

Todo: run an experiment with 100+ tweets to see to what extent the
classification works. Further this has to be compared to a control set, of
manually classified tweets.  

\subsection{Features addition to tweets.}
\subsection{Labeled graph propagation.}
\subsection{Combination of the three above}
\subsection{Trend creation}
\subsection{Comparrison}
