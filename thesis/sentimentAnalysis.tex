
\chapter{Sentiment Classification}
This section describes the experiments done. High level description and
execution of experiments. Detailed execution and technical details in appendix. 

#TODO what is sentiment
#TODO why do we get it
#TODO how do we use it 

% todo figure out what experiments are smart to do. 

* Experiment with the time frame of the prediction of the trend. 
	* Typically the variation of time. What's the longest into the future that
we can predict the trend?

\section{Word count classification}
#TODO description of method. 
Comparison of positive vs negative words.

A basic approach to classification is to count words. Positive and negative
words. To do this it is necessary have dictionaries to provide the
classification of simple words. Section:\ref{data:dictionaries} describes the
dictionaries used. Some of them are quite simple, while others are more
extensive. 

#TODO calculating the polarity.  
The polarity of a given tweet is based on the difference in the amount of
positive verses negative words.

pos/totw - neg/totw 

#TODO write about the threshold variations.  

#TODO write results from the classification of the different dictionaries.

#TODO error analysis of the discarded words:
duplicate-words-from-monogram-compilation.txt

#TODO write about drawbacks.
\paragraph{Word positioning}
The dictionaries are based on the manually labeled
tweets, so we can't create bi and tri-grams based on the position of a word in a tweet.
Rather there is no way of automatically decide if a single word is positive or
negative. 

\section{With Classifiers}

#TODO write about drawbacks. 

\subsection{SVM}
With both datasets.
Using the self compiled monogram dictionaries. 

Results from svm testing. which kernel works best?

\subsection{Naive Bayes}
With both datasets.
Using the self compiled monogram dictionaries. 

Results from testing with different dictionaries. 

\section{Comparison of classifiers}
#TODO highlights of the classifiers
#TODO common denominators. commonalities.   
#TODO comparing the results of the classifiers.  

\section{Comments}
#TODO improvements
#TODO drawbacks
#TODO future work

\section{Conclusions}
#TODO summarize the stuff we have learned shortly. 
#TODO mention future work. 
