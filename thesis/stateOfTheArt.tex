\chapter{Background and Previous Work}\label{previous_work}
Previous work and the background information needed for the context of this
thesis is described in this chapter. We look at how microblogging activity is
can be related to finance \cite{bollen2011}. And how sentiment can be used to
predict stock markets.

This chapter contains related research, context knowledge and introduction to
the subjects discussed in this thesis. Twitter as a microblogging platform is
introduced in \ref{previous_work:twitter}. Section
\ref{previous_work:sentiment}, describes sentiment and what it can be used for.
Finance and trading as a context is described in section
\ref{previous_work:finance}. And last trends are talked about in section
\ref{previous_work:trends}.

\section{Twitter}\label{previous_work:twitter}
Twitter is a social informations network. 
It's a real-time service for sharing and gathering small messages. These
messages can represent everything from a persons opinion of ice cream, to the
latest changes in the financial market or pictures from a Mars rover. 

At the core of Twitter you have the Tweet. The Tweet is the 140 character
message. 
Tweets lets you communicate with other users, share photos and post all kinds of
information. The small size of the tweets are not a hindrance for the flow of
information. 
\footnote{About Twitter: \url{https://twitter.com/about}}

The fast growing messaging service handles 1.6 billion search queries every day.
In 2012 the 500 million users would generate 3.2 queries each day. 340 million tweets were posted every day. 
\footnote{Wikipedia: \url{http://en.wikipedia.org/wiki/Twitter}} 

Most medium and large companies have a presence on Twitter today. Posts can contain
any type of information, from promotional content to service status to
financial reports. \cite[p8]{annikajubbega11:twitter_driver_stock_price} says
that 77 of the Fortune 100 companies have a twitter account. 

Companies use Twitter to communicate with customers. Customers can post
questions and feedback, while the company posts answers and information.
Questions can be asked with a specific hashtag. Or with an at sign to target a
specific user. This makes it easy to filter the messages, and therefore easier
to get in contact with the customer. Best Buy demonstrated the successfulness of
twitter in customer relations by answering questions with a specific hashtag. In
2009 they had answered nearly 20 thousand questions using twitter.
\cite[p1]{Li2013206} Market Intelligence is also a major aspect of the
microblogging sphere.

Twitter represents one of the largest and most dynamic datasets of user
generated content. Along with Facebook twitter data is in real time. This has major
implications for anyone who are interested in sentiment, public opinion or
customer interaction. \cite[]{sperious11}

A typical tweet contains about 11 words and provides an opinion or state of
mind or a piece of information. Tweets can contain hashtags: '\#something', user:
'@username', or other adaptations of prefixes such as '\$STO' which represents a
stock. The different prefixes or tags (\$, \#, @) easily distinguishes the
content of the tweet. This also makes it easier to search and classify the
content of tweets. Figure:\ref{fig:sto} and figure:\ref{fig:tweet1} are
examples of tweets as shown on twitter.

The retrieval of tweets seems like a challenge. But Twitter has made this easy
by providing an API \footnote{API: Application programming interface}. With the
API you can write tweets and update the status of a user. But the best part of
the API is that it provides search capabilities. To get a certain subset of all
tweets, we can use the search function and view only the tweets we want. 

On the front page of twitter we have the search function at the top right of
the page. The search provides the ability to specify which types of tweets you
want. And gives you the opportunity to find the information you are looking for. 

\begin{figure}[htb]
    \centering
    \includegraphics[width=\textwidth]{STO} 
    \caption{Typical tweet from Twitter.}
    \label{fig:sto}
\end{figure}

%\begin{figure}[htb]
%    \centering
%    \includegraphics[width=\textwidth]{STO2} 
%    \caption{The text that shows under the image, image text.}
%    \label{fig:sto2}
%\end{figure}

\begin{figure}[htb]
    \centering
    \includegraphics[width=\textwidth]{tweet1} 
    \caption{Typical tweet from Twitter.}
    \label{fig:tweet1}
\end{figure}

%\begin{figure}[htb]
%    \centering
%    \includegraphics[width=\textwidth]{tweet2} 
%    \caption{The text that shows under the image, image text.}
%    \label{fig:tweet2}
%\end{figure}
%
%\begin{figure}[htb]
%    \centering
%    \includegraphics[width=\textwidth]{tweet3} 
%    \caption{The text that shows under the image, image text.}
%    \label{fig:tweet3}
%\end{figure}

\section{Sentiment}\label{previous_work:sentiment}

Opinion mining on the web is not new. In resent years it has
become attractive to traders. 
The use of Twitter and social media is increasing in increasing. Both in
business and with private users.
This means a surplus of raw data with
easy access. Companies all over the world has started to use the social
networks to their benefit. The use of information from social media has become
part of the trend, although there are some drawbacks and shortcomings. Noise and
garbage is one of them. Even if you're right 80\% of the time, the last 20\% can
prove devastating. \cite[]{stevenson12:social_media_stock_pickers}

Sentiment broadly refers to a persons state of mind. Based on the opinion the
person will do optimistic or pessimistic choices. A positive mindset leads to
optimistic judgements of future events, and a negative state of mind leads to
pessimistic choices.
\cite[p4]{doukas10:sentiment_and_momentum}

The users may have different roles and intentions in different
communities in the microblogging sphere, \cite[]{java07}. 
A users intentions and its reasons for participation might be a factor in the sentiment analysis.

\subsection{What is Sentiment Analysis}
There are two main categories of approaches to sentiment analysis. 
	The first is to use a classifier. The classifier can use methods such as
naive Byes, maximum entropy or support vector model \cite[]{Li2013206} This is
typically a method where it would be natural to use machine learning of
evolutionary algorithms to increase the classification correctness over time. 
	The other is to use linguistic resources, such as corpora of negative and
positive words. The developed linguistic resources are used to classify the
sentiment of the text \cite[]{Li2013206}.

Li and Li has created a framework for sentiment analysis. The system
consists of four main steps  and is tested with experiments on twitter. 
	First they do topic detection, identifying and extracting the topics
mentioned in the tweet. 
	Secondly opinions are classified. The polarity of the opinion is decided and
the users impression is captured. 	
	Third. Credibility is assessed. This creates a better summarization of the
expresser's credibility. 
	Fourth, step one, two, and three is aggregated to reflect the true opinion
and point of view.
	Combining the first three steps in the fourth results in a truer reflection of
the expresser's opinion. \cite[]{Li2013206} 

One way of classifying tweets is to use predefined lexicon of positive and
negative words. Consumer confidence and fluctuations of voting polls can be
tracked in this way \cite[]{connor2010}.

The work of \cite[]{diakopoulos2010} describes a methodology for better
understanding of temporal dynamics of sentiment. 
The system uses visual representation to achieve this. 
This is investigated in the reaction to debate video.
	Further \cite[]{diakopoulos2010} detects sentiment pulse and controversial
topics with the help of visualisation and metrics. 
	\cite[]{diakopoulos2010} used crowdsourcing\footnote{Crowdsourcing is the
practice of obtaining needed services, ideas, or content by soliciting
contributions from a large group of people, and especially from an online
community, rather than from traditional employees or suppliers.
\url{http://en.wikipedia.org/wiki/Crowdsourcing}} to classify batches of tweets.
This was accomplished with Amazon Mechanical Turk, a crowdsourcing
site\footnote{Amazon Mechanical Turk (AMT): \url{https://www.mturk.com/mturk/}}.

\cite[]{barbosa10} explores the problem of noise in biased and noisy data. 
They focus on noisy labels and add features to the tweets to increase the
classification properties of the tweets. To filter out tweets that don't project a
sentiment tweets are classified as subjective or objective. The subjective tweets are classified as positive or negative.

Classification of tweets can be generalised by using features. Features are
elements such as unigrams, bigrams, and part-of-speech tags. An abstract
representation of a tweet would be beneficiary to the classification. In this
abstract representation \cite[]{barbosa10} propose to use characteristics about
how tweets are written and meta-information about the words in tweets. 
Meta-features and tweet syntax features are further features that can improve
classification. Meta-features are information about the tweet, such as
location, language, and number of retweets. The tweet syntax features are things such as hashtags, retweet,
reply, links, punctuation and emoticons \cite[]{barbosa10}. 

Another approach to the sentiment challenges with twitter is explored by
\cite[]{becker13}. They explore techniques for contextual polarity
disambiguation and message polarity classification. Constrained and supervised
learning is used to create models for classification. They describe a system
that solves these tasks with the help of polarity lexicons and dependency
parsers. Expanded vocabulary is one of the main aspects of their success, as
they say in their findings: "We hypothesize this performance is largely due to
the expanded vocabulary obtained via unlabeled data and the richer syntactic
context captured with dependency path representations." \cite[]{becker13}

In contrast to \cite[]{becker13}, \cite[]{sperious11} has used distant
supervision and labeled propagation on a graph based data structure. The data
structure represents users with tweets as nodes. And tweets with bigrams,
unigrams, hashtags, etc as subnodes of the tweets. A label propagation approach
rivals a model supervised with in-domain annotated tweets and outperforms the
noisily supervised classifier and a lexicon-based polarity ratio classifier.
\cite[]{sperious11} 

\subsection{Sentiment analysis in Finance}
\cite[p2]{Brown20041} writes the following on over-reaction of investors: 
"\textit{He(Siegel (1992)) concludes that shifts in investor sentiment are correlated
with market returns around the crash. Intuitively, sentiment represents the
expectations of market participants relative to a norm: a bullish (bearish)
investor expects returns to be above (below) average, whatever ‘‘average’’ may
be."}. 
In the light of resent changes in the financial world and the use
of sentiment from social media, the notion that opinions and sentiment of
investors and market actors affect the market is not a new observation.

Use of sentiment can potentially predict changes and trends in the market.
Bad news in an optimistic period creates cognitive dissonance in the small
investors. This impacts the market by slowing down the selling rate of loosing
stocks. \cite[p29]{doukas10:sentiment_and_momentum}
Further we can see that optimistic sentiment has a 2\% monthly average return.
While the investor sentiment is pessimistic we see a drastic reduction in
returns. Down to 0.34\%,\cite[p5]{doukas10:sentiment_and_momentum}.
After optimistic periods it is indicated that the monthly return is reduced to
-0.49\%. On the contrary there is no equivalent change after a pessimistic
period, \cite[p6-7]{doukas10:sentiment_and_momentum}.
Momentum profits are only significant when the sentiment is optimistic,
\cite[p29]{doukas10:sentiment_and_momentum}.

Hope and fear is used by \cite[]{Zhang201155} to decide the movement of the
market. The sentiment is aggregated to be hopeful or fearful. This basically
focuses on positivity and negativity of the sentiment of that particular day.
The daily sentiment is then compared to the market indicators of the same day
to create a prediction of the market. \cite[]{Zhang201155} finds that calm
times give little hope or other emotions. Little turmoil results in few
fluctuations in the market. And opposite, lots of emotions(hope, worry, fear),
gives speed to the market.

\cite[p3]{Brown20041} indicates that the sentiment does not cause subsequent
market returns. For a short-term marketing timing this is bad news. However
with the changes in social media over the last decade how is the situation
today? With the microblogging sphere of today we can easily see the
correlation of sentiment and the market indicators,
\cite[]{annikajubbega11:twitter_driver_stock_price}. But
does the sentiment cause changes in the market-return?
\cite[p3]{Brown20041} also says that optimism is associated with overvaluation
and subsequent low returns.

\cite[p]{Brown20041} concludes that aggregated sentiment measures has strong
co-movement with changes in the market. He also indicates that sentiment
doesn't appear to be a good trading strategy. This, in the view of
\cite[]{Zhang201155}, indicates a leap in sentiment research and what is possible
with the microblogging of today.

\section{Finance and Trading}\label{previous_work:finance}
%Finance and Trading on and with twitter. 
%\cite[p.2]{annikajubbega11:twitter_driver_stock_price}

The management of assets or liabilities and the management of funds over a
period of time is called Finance. In finance the valuation of assets are time
dependant. The same asset is not worth the same now and in a few minutes. Assets
are priced based on expected returns and risk level. The three sub categories
of finance are: personal, corporate and public \footnote{Wikipedia:\url{http://en.wikipedia.org/wiki/Finance}}.
These categories describes very different parts of the financial world. 

Trading is the action of buying or selling financial instruments.
Financial instruments can be stocks, bonds, derivatives or commodities 
\footnote{Wikipedia:\url{http://en.wikipedia.org/wiki/Trader_(finance)}}.
Trades takes place in markets, stock markets, derivatives markets or commodity
markets.

A new aspect to trading the last decade has been online trading
\footnote{Wikipedia: \url{https://en.wikipedia.org/wiki/Online_trading}}.
Speed, ease of use, and low costs made the online brokers popular. Many brokers
provide platforms for trade and analysis to select potential investments.  

The tools provided are often some form of technical analysis \footnote{Wikipedia:
\url{http://en.wikipedia.org/wiki/Technical_analysis}}. Technical analysis is
the study of past market data. Mostly volume and price. The purpose of technical
analysis is to forecast the direction of prices.  

Among tools and techniques in technical analysis are charts, market indicators,
moving average, and relative strength index. It is also quite common to combine
techniques to acquire better predictions. When looking at put/call rations,
short interest, implied volatility, and bull/bear indicators of sentiment is
also important to take into consideration. 

Technical analysis focus at numeric values, such as volume and price, where fundamental analysis \footnote{Wikipedia:
\url{http://en.wikipedia.org/wiki/Fundamental_analysis}} analyse company health,
financial statements, production rates, earnings, management, and competitive
advantages. Day traders and pit traders prefer technical analysis over
financial analysis.   

Two principles of technical analysis are, 'History repeats itself', and
'Prices move in trends'. 'History repeats itself' refers to the belief that
traders will do the same actions again and again. Technicians believe that the
repeated behaviour can be recognized as a pattern and be observed on a chart.
'Prices move in trends' is the belief that the price of a commodity will move
directionally over a period of time. Relative highs and relative lows are
indicators of a trend. Consecutive lower highs indicates a downward trend.  

An interesting aspect of trading and finance is the behavioral
one\footnote{Wikipedia:
\url{https://en.wikipedia.org/wiki/Behavioral_finance}}.
Behavioral finance is the field of research that study the effects of
cognitive, social, emotional and psychological factors economic decisions. It
also includes the consequences of resource allocation, market price and
returns. 
%

\section{The Trend}\label{previous_work:trends}
The trend is the general opinion of the masses. As defined by the Free
Dictionary:  
"The direction and momentum of a market, price, economy, or other measure. For
example, if the price of a security is going mainly downward with only a few
gains, it is said to be on a downward trend. Identifying and
predicting trends is important finding the right moment to buy and sell
securities. Trends are especially important in technical analysis, which
recommends buying at the bottom of a downward trend and selling at the top of an
upward trend."
\footnote{Dictionary description of trend: \url{http://financial-dictionary.thefreedictionary.com/Trend}}

Trends work in much the same way as opinions. When an opinion or concept is
presented to others they start to think the same thing or feel the same way or
move in the same direction. The first group of people that move in the same
direction are called trend setters. They are the people that show others how
the trend works and what this trend is about. 

On twitter we have lots of subcultures that all express themselves on their
specific topic. It can be technology, art, finance, or fashion among others.  
In the sense of twitter we can take a step back and look at the content of
messages and from there see if we can find common topics that people talk
about, this being the topic of a subculture or a subspace of twitter. To get
the trend we have to look at the content of the messages in a subspace. Given
that a trend is the combined general opinion of a group. We can analyse the
group and see if we can find certain topics or areas of interest that aggregates
to a trend.  

Stock market prediction
\footnote{Wikipedia:\url{http://en.wikipedia.org/wiki/Stock_market_prediction}}
is the act of trying to predict or determine the future
value of a stock. A way to do this is to look for trends in the data. The trend
is a tendency of movement in a particular direction for a financial market.

There are three categories of trends in finance. Primary, secondary and secular
trends. Where primary trends have a medium time frame, secondary have a short
time frame, and secular has long time frames \footnote{Wikipedia:
\url{http://en.wikipedia.org/wiki/Market_trend}}. Bull market and bear market
are terms that, respectively, describe upward and downward market trends. 
Trends are often found by using technical analysis.  

Secular trends are trends that last between 5 and 25 years. A secular bull
market consists of many larger bull markets and many smaller bear markets.
Primary trends last a year or more. We can also observe market tops and
bottoms here. These are trend reversal points. Secondary trends has a
duration of a few weeks or months. The secondary market trend is change in
price direction within a primary trend. The small changes are often called
market corrections. The short term correction is often between 5 and 20 percent.

When looking for usage of twitter trends, we find little to confirm research in
this area. No material or indication is found to suggest that trending on twitter
is researched in regards to sentiment analysis of tweets and finance. 

%
